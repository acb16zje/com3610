\documentclass[12pt, a4paper]{report}
\edef\restoreparindent{\parindent=\the\parindent\relax}
\usepackage[UKenglish]{babel}
\usepackage[bibstyle=ieee, sorting=nty]{biblatex}
\usepackage[labelfont=bf]{caption}
\usepackage{csquotes}
\usepackage{fancyhdr}
\usepackage{float}
\usepackage[bottom]{footmisc}
\usepackage{graphicx}
\usepackage[hidelinks]{hyperref}
\usepackage{parskip}

\linespread{1.2}
\restoreparindent

\pagestyle{fancy}
\fancyhf{}
\fancyhead[C]{\leftmark}
\fancyfoot[C]{\thepage}

\addbibresource{references.bib}

\begin{document}
\begin{titlepage}
	\centering
	\includegraphics[width=10cm]{tuos_logo}\par\vspace{1cm}
	\vspace{1cm}

	{\huge\bfseries Finding Security Issues in (Open Source) Software Repositories\par}
	\vspace{1cm}

	{\Large Zer Jun Eng\par}
	\vspace{1cm}

	supervised by\par Dr.~Achim \textsc{Brucker}
	\vfill

	{This report is submitted in partial fulfilment of the requirement for the degree of MEng Software
		Enginnering by Zer Jun Eng}
	\vfill

	{\large COM3610}
	\vfill

	{\large \today}
\end{titlepage}

\pagenumbering{roman}

\chapter*{Declaration}
\addcontentsline{toc}{chapter}{Declaration}
All sentences or passages quoted in this report from other people's work have been specifically
acknowledged by clear cross-referencing to author, work and page(s). Any illustrations that are not
the work of the author of this report have been used with the explicit permission of the originator
and are specifically acknowledged. I understand that failure to do this amounts to plagiarism and
will be considered grounds for failure in this project and the degree examination as a whole.
\vspace{2cm}

\noindent \begin{tabular}{llp{4.5cm}}
	Name & : & Zer Jun Eng \\ \cline{3-3}
	\\ [-0.5em]
	Date & : & \today      \\ \cline{3-3}
\end{tabular}

\newpage

% \chapter*{Abstract} \addcontentsline{toc}{chapter}{Abstract}

% \chapter*{Acknowledgements} \addcontentsline{toc}{chapter}{Acknowledgements} I would like to thank
% my family for...

% \newpage

\tableofcontents

% \listoffigures \addcontentsline{toc}{chapter}{List of Figures}

% \listoftables \addcontentsline{toc}{chapter}{List of Tables}

\newpage

\pagenumbering{arabic}

\chapter{Introduction}
\section{Background}
Free/Libre and Open Source Software (\textbf{FLOSS}) is a type of software which license allows the
users to inspect, use, modify and redistribute the software's source code \cite{crowston_2012}.
Since the introduction of Git, and later the Git repositories hosting site such as GitHub, many
users have started to make their softwares open source by storing them as public repositories on
GitHub. As a result, the participation of global communities into \textbf{FLOSS} projects has
started to grow and different contributions were made to improve the softwares quality, which
included fixing the software vulnerabilities \cite{dabbish_2012}.

Building a secure software is expensive, difficult, and time-consuming. In \textbf{FLOSS} projects,
it is necessary to know when and how a security vulnerability is fixed. Therefore, having a list of
changelogs or informative git commit messages that record the fixed security vulnerabilities is
helpful. However, Arora and Telang \cite{arora_2005} stated that some open source developers believe
that public disclosure of security vulnerabilities patch is dangerous, and thus vulnerability fixing
commits are not commonly identified in some open source software repositories to prevent malicious
exploits. In this case, a repository mining tool that investigate vulnerability patterns and
identify vulnerable software components can be developed to reduce the time and cost required to
mitigate the vulnerabilities.

\section{Objectives}
\label{sec:objectives}
\begin{itemize}
	\item Identify the security patterns of the most popular security issues in OWASP Top Ten Project
	\cite{owasp_2017}. The patterns should be expressed using regular expressions.
	\item Develop a repository mining tool to search through the commit history of a repository and
	find a list of commit messages that match the patterns.
	\item Extend the mining tool to run an evaluation process which checks the code difference in the
	commits found, where the added lines represent the fix for a vulnerability and the deleted lines represent the vulnerability.
\end{itemize}

\section{Challenges}
\begin{itemize}
	\item \textbf{Data}: There are a large numbers of open source repositories available on GitHub.
	However, it is challenging to find a set of sample repositories that can produce accurate and
	consistent results.
	\item \textbf{Evaluation}: After mining a list of commit messages that contain the identified
	patterns, the evaluation process might not correctly locate the lines of code that addressed the
	security vulnerability.
	\item \textbf{Time}: Large repository such as Linux which has more than 780,000 commits in total
	\cite{linux_repo} could be extremely time-consuming for the repository mining tool to complete the
	search and evaluation process.
\end{itemize}

\section{Report Structure}
\textbf{Chapter 2} reviews a range of academic articles, theories and previous studies that
is related to this project, as well as investigating the techniques and tools to be used.

\noindent\textbf{Chapter 3} is a list of detailed requirements and a thorough analysis for design,
implementation and testing stage.

\noindent\textbf{Chapter 4} is a comparison between different design concepts, where the advantages
and disadvantages of difference approach are stated. The chosen design is justified with suitable
diagrams provided including wireframes and UML.

\noindent\textbf{Chapter 5} describes the implementation process by highlighting novel aspects to
the algorithms used. Testing are performed by following a suitable model to evaluate the
implementation.

\noindent\textbf{Chapter 6} presents all the results along with critical discussions about the main
findings,	and outlines the possible improvements that could be made in the future work.

\noindent\textbf{Chapter 7} summarises the main points of previous chapters and emphasise the
results found.

\section{Relationship to Degree Programme}
This project focusses on researching real-world software security problem by deploying a repository
mining tool to open source software repositories with the purpose of studying the patterns of
different security vulnerabilities patch. This relates to the 'Software Engineering' degree as it
requires a good understanding in version control system and it aims to improve softwares quality by
reducing the time and effort needed to locate and fix security vulnerabilities in the source code.

\chapter{Analysis} % Discussion of problems to be solved and possible techniques and tools The
purpose of this chapter is to discuss the problems to be solved and consider some of the core
decisions to be made before starting the implementation.

\section{Problems}
As mentioned in \hyperref[sec:objectives]{\textbf{Section 1.2}}, the repository mining tool must be
able to detect commits that contain distinct patterns such as \textit{fix}, \textit{patch},
\textit{vulnerability} etc. After extracting a possible list of commits, it should perform an
evaluation process to identify the actual commits that fixed security vulnerability.	This could be
hard because not all open source software repositories are using the same programming language.
Hence, it might be difficult to determine the actual lines of code that addressed the
vulnerabilities.

\section{Tools}
\begin{itemize}
	\item PyGithub is a Python library build to access the GitHub API \cite{pygithub}.
	\item GitPython is a Python library build to interact with Git repositories using a combination of
	python and git command implementation \cite{gitpython}.
	\item Secbench Mining Tool is a repository mining tool build by The Quasar Research Group to mine
	vulnerability patterns from GitHub repositories \cite{secbench}.
\end{itemize}

\section{Plan of Action}
\begin{itemize}
	\item \textbf{Week 1}: Discuss about the problems encountered when writing the description stage
	with supervisor. Ask supervisor for feedback if available.
	\item \textbf{Week 2}: Should have finished the draft of description stage by the start of this
	week. Discuss the draft with supervisor to check for mistakes. Research work should be started
	during this week.
	\item \textbf{Week 3}: Should have found at least 5 sources, which is related to the project. Plan
	the outline and start writing the introductory section for literature review.
	\item \textbf{Week 4}: Perform a source analysis and note down the sentences that is related to
	this project. Started the research for suitable techniques and tools to be used in this project.
	\item \textbf{Week 5}: Organise the sources and start writing the literature review.
	\item \textbf{Week 6}: Finish the literature review and start writing the requirements and
	analysis.
	\item \textbf{Week 7}: Finish the requirements and analysis. Proof read the document and write the
	abstract.
	\item \textbf{Week 8}: Submit the first draft of survey and analysis to supervisor to seek early
	feedback.
	\item \textbf{Week 9}: Amend the document based on the feedback.
	\item \textbf{Week 10}: Ask the supervisor about any final changes.
	\item \textbf{Week 11}: Survey and analysis stage should be completed and ready to submit. Discuss
	with supervisor about the work to do during the holiday.
\end{itemize}

\chapter{Literature Review}
This chapter will explore the previous and existing works related to this project, as well as
reviewing the security issues that might exist in an open source Git repository.

\section{Software Security}
There are currently two approaches to the license distribution of software: open source and closed
source. The users of closed source software are limited to accept the level of security
provided by their chosen vendor. In contrast, open source softwares provide more flexibility and
freedom over the security option to their users \cite{payne_2002}, where the users can decide to
wait for a patch from the vendor or collaborate with the community to develop their own. In this
project, it is worth investigating that whether the security issues exis

Hoepman and Jacobs \cite{hoepman_2007} suggested that open source softwares will have better
security and reliability than closed source softwares through the power of open data and
crowdsourcing. Conversely, Schryen \cite{schryen_2011} has shown that open source and closed source
softwares do not have significant difference in terms of security vulnerabilities in his experiment.
However, Cowan \cite{cowan_2003} stated that both statements are not totally true, which he
indicated that the security of a software is largely determined by the developers but not the source
availability model.

\section{Security Issues in Open Source Sotware Repositories}
\subsection{OWASP}

\section{foo}
\begin{table}[H]
	\begin{center}
		\begin{tabular}{ c c c }
			cell1 & cell2 & cell3 \\
      cell4 & cell5 & cell6 \\
			cell7 & cell8 & cell9
		\end{tabular}
		\captionof{table}{cell table} \label{table:celltable}
	\end{center}
\end{table}

According to \hyperref[table:celltable]{\textbf{Table 3.1}}, \ref{table:celltable}

% \chapter{Requirements and Analysis}

% \chapter{Design}

% \chapter{Implementation and Testing}

% \chapter{Results and Discussion}

% \chapter{Conclusion}

\printbibliography[heading=bibintoc]

\end{document}