\documentclass[12pt, a4paper]{report}
\edef\restoreparindent{\parindent=\the\parindent\relax}
\usepackage[UKenglish]{babel}
\usepackage[bibstyle=ieee, sorting=nty]{biblatex}
\usepackage[labelfont=bf]{caption}
\usepackage{csquotes}
\usepackage{fancyhdr}
\usepackage{float}
\usepackage[bottom]{footmisc}
\usepackage{graphicx}
\usepackage[hidelinks]{hyperref}
\usepackage{parskip}
\usepackage{pgfgantt}

\linespread{1.2}
\restoreparindent

\pagestyle{fancy}
\fancyhf{}
\fancyhead[C]{\leftmark}
\fancyfoot[C]{\thepage}

\addbibresource{references.bib}

\begin{document}
\begin{titlepage}
	\centering
	\includegraphics[width=10cm]{tuos_logo}\par\vspace{1cm}
	\vspace{1cm}

	{\huge\bfseries Finding Security Issues in (Open Source) Software Repositories\par}
	\vspace{1cm}

	{\Large Zer Jun Eng\par}
	\vspace{1cm}

	supervised by\par Dr.~Achim \textsc{Brucker}
	\vfill

	{This report is submitted in partial fulfilment of the requirement for the degree of MEng Software
		Enginnering by Zer Jun Eng}
	\vfill

	{\large COM3610}
	\vfill

	{\large \today}
\end{titlepage}

\pagenumbering{roman}

\chapter*{Declaration}
\addcontentsline{toc}{chapter}{Declaration}
All sentences or passages quoted in this report from other people's work have been specifically
acknowledged by clear cross-referencing to author, work and page(s). Any illustrations that are not
the work of the author of this report have been used with the explicit permission of the originator
and are specifically acknowledged. I understand that failure to do this amounts to plagiarism and
will be considered grounds for failure in this project and the degree examination as a whole.
\vspace{2cm}

\noindent \begin{tabular}{llp{4.5cm}}
	Name & : & Zer Jun Eng \\ \cline{3-3}
	\\ [-0.5em]
	Date & : & \today      \\ \cline{3-3}
\end{tabular}

\newpage

% \chapter*{Abstract}
% \addcontentsline{toc}{chapter}{Abstract}

\chapter*{Acknowledgements}
\addcontentsline{toc}{chapter}{Acknowledgements}
I would firstly like to thank my parents for their unconditional love and the full financial supoprt
throughout my university life. It would not be possible for me to finish this project and my course
without them.

I would also like to thank my supervisor, Dr. Achim Brucker, who are continuously providing
constructive advice for my project. I am honoured to work with you, and I look forward for more
working opportunities with you in the future.

Finally, I would like to thank my friends: Jia Hua, Wei Kin, Justin and Grace, who have spent
countless sleepless night with me in the Diamond for the past two years. It was a truly memorable
and unforgettable experience. I am glad to have them as my friends.

\newpage

\tableofcontents

\listoffigures \addcontentsline{toc}{chapter}{List of Figures}

\listoftables \addcontentsline{toc}{chapter}{List of Tables}

\newpage

\pagenumbering{arabic}

\chapter{Introduction}
\section{Background}
Free/Libre and Open Source Software (\textbf{FLOSS}) is a type of software which its license allows
the users to inspect, use, modify and redistribute the software's source code \cite{crowston_2012}.
Since the introduction of Git, and later the Git repositories hosting site such as GitHub, many
users have started to make their softwares open source by storing them as public repositories on
GitHub. As a result, the participation of global communities into \textbf{FLOSS} projects has
started to grow and different contributions were made to improve the softwares quality, which
included fixing the software vulnerabilities \cite{dabbish_2012}.

Building a secure software is expensive, difficult, and time-consuming. In \textbf{FLOSS} projects,
it is necessary to know when and how a security vulnerability is fixed. Therefore, having a list of
changelogs or informative git commit messages that record the fixed security vulnerabilities is
helpful. However, Arora and Telang \cite{arora_2005} stated that some open source developers believe
that public disclosure of security vulnerabilities patch is dangerous, and thus vulnerability fixing
commits are not commonly identified in some open source software repositories to prevent malicious
exploits. Hence, a repository mining tool that investigate vulnerability patterns and identify
vulnerable software components can be developed to reduce the time and cost required to mitigate the
vulnerabilities.

\section{Objectives}
\label{sec:objectives}
\begin{itemize}
	\item Identify the security patterns of the most popular security issues in OWASP Top Ten Project.
	The patterns should be expressed using regular expressions.
	\item Develop a repository mining tool to search through the commit history of a repository and
	find a list of commit messages that match the patterns. The list should be produced in a suitable
	file format such as JSON, XML, or CSV.
	\item Extend the mining tool which checks the code difference in the commits found to obtain the
	actual commits fixing the security vulnerabilities. This extension should separate from the mining
	process to make the mining results easier to verify and debug.
\end{itemize}

\section{Challenges}
\begin{itemize}
	\item \textbf{Data}: There are a large numbers of open source repositories available on GitHub.
	However, it is challenging to find a set of sample repositories that can produce accurate and
	consistent results.
	\item \textbf{Misclassification}: Commit messages for a same vulnerability patch are not always
	the same, thus misclassification of commit messages is inevitable. Using regular expressions to
	match the patterns in the mining process does not guarantee correctness of the result.
	\item \textbf{Evaluation}: After mining a list of commits that contain the identified patterns in
	its message, the evaluation process might not correctly locate the lines of code that addressed
	the security vulnerability. It might be required to perform manual evaluation to correctly
	identify some of the results.
	\item \textbf{Time}: Large repository such as Linux which has more than 780,000 commits in total
	\cite{linux_repo} could be extremely time-consuming for the repository mining tool to complete the
	search and evaluation process.
\end{itemize}

\section{Report Structure}
\textbf{Chapter 2} reviews a range of academic articles, theories and previous studies that
is related to this project, as well as investigating the techniques and tools to be used.

\noindent\textbf{Chapter 3} is a list of detailed requirements and a thorough analysis for design,
implementation and testing stage.

\noindent\textbf{Chapter 4} is a comparison between different design concepts, where the advantages
and disadvantages of difference approach are stated. The chosen design is justified with suitable
diagrams provided including wireframes and UML.

\noindent\textbf{Chapter 5} describes the implementation process by highlighting novel aspects to
the algorithms used. Testing are performed by following a suitable model to evaluate the
implementation.

\noindent\textbf{Chapter 6} presents all the results along with critical discussions about the main
findings,	and outlines the possible improvements that could be made in the future work.

\noindent\textbf{Chapter 7} summarises the main points of previous chapters and emphasise the
results found.

\section{Relationship to Degree Programme}
This project will be focused on researching real-world software security problem by deploying a
repository mining tool to open source software repositories with the purpose of studying the
patterns of different security vulnerabilities patch. This relates to the 'Software Engineering'
degree as it requires a good understanding in version control system and it aims to improve
softwares quality by reducing the time and effort needed to locate and fix security vulnerabilities
in the source code.

\chapter{Literature Review}
This chapter will start with the background contents of the project, and then focus on discussing
the security aspect of open source softwares. Lastly, previous and existing relevant work are
reviewed and a critical analysis is provided for the comparison of these resources and this project.

\section{Open Source Security}
There are currently two approaches to the license distribution of software: open source and closed
source. The users of closed source software are limited to accept the level of security
provided by their chosen vendor. In contrast, open source softwares provide more flexibility and
freedom over the security option to their users \cite{payne_2002}, where the users can decide to
wait for a patch from the vendor or collaborate with the community to develop their own.

Hoepman and Jacobs \cite{hoepman_2007} suggested that open source softwares will have better
security and reliability than closed source softwares through the power of open data and
crowdsourcing. Conversely, Schryen \cite{schryen_2011} has shown that open source and closed source
softwares do not have significant difference in terms of security vulnerabilities in his experiment
and concluded that the policy of the developers is the main factor that determines the security.
Wheeler \cite{wheeler_2015} agrees with Schryen's view, but he also pointed out that open source
systems are more resistant to attacks based on his researches.

While both statements might be true, Cowan \cite{cowan_2003} indicates that there are many factors
determining the security of a software, and the source availability model is not the primary factor.
Othmane et al. \cite{othmane_2015} have classified the main factors that affects the time required
to fix a vulnerability into different categories, and one of them that is related to this project is
the vulnerabilities characteristics. Vulnerabilities characteristics is the categorisation of
security attacks into different types, which will be discussed in the next section.

\section{Security Issues in Open Source Softwares}
The Open Web Application Security Project (\textbf{OWASP}) is a worldwide non-profit organization
committed to improve and raise the awareness of software security in the open source community
\cite{owasp_home}. The project members of \textbf{OWASP} have worked together to produce a list of
the most critical web application security risks based on the community feedback and comprehensive
data contributed by different organizations. The list consists of ten categories of security attacks
which are considered to be the most dangerous and popular in the recent years. The list published by
\textbf{OWASP} in 2017 \cite{owasp_top10} will be analysed and the security risks listed in older
versions of the top ten project will covered in this section too.

\subsection{Injection}
An injection attack is the exploitation of a software vulnerability where the attacker injects
malicious code into the software and perform harmful executions. The most common types of injection
attacks are Structured Query Language (\textbf{SQL}) injection, cross-site scripting (\textbf{XSS}),
and command injection \cite{pietraszek_2006}. Injection flaws are very widespread and easy to
discover when attackers have access to the source code, where they could use a code scanner tool to
find all possible ways of the injection attacks.

\subsection{Broken Authentication}
Broken authentication happens when the attackers are allowed to perform malicious actions such as
brute force dictionary attack on the authentication system. It might also be one of the outcome of a
successful injection attack. According to Huluka and Popov \cite{huluka_2012}, this vulnerability
are very prevalent and has various causes, in which the lack of attention to security details is the
most critial because developers often overlook certain scenarios which are likely to be exploited by
attackers.

\subsection{Using Componenets with Known Vulnerabilities}
Components such as plugins, libraries, and other modules can be found in most of the softwares.
Using components could reduce the amount of work and time required to develop a software. However,
these components are very likely to be maintained by different developers or organizations.

\section{Related Work}


% \section{foo}
% \begin{table}[H]
% 	\begin{center}
% 		\begin{tabular}{ c c c }
% 			cell1 & cell2 & cell3 \\
%       cell4 & cell5 & cell6 \\
% 			cell7 & cell8 & cell9
% 		\end{tabular}
% 		\captionof{table}{cell table} \label{table:celltable}
% 	\end{center}
% \end{table}

% According to \hyperref[table:celltable]{\textbf{Table 3.1}}, \ref{table:celltable}

\chapter{Requirements and Analysis}
The purpose of this chapter is to discuss the problems to be solved and consider some of the core
decisions to be made before starting the implementation.

\section{Problems}
As mentioned in \hyperref[sec:objectives]{\textbf{Section 1.2}}, the repository mining tool must be
able to detect commits that contain distinct patterns such as \textit{fix}, \textit{patch},
\textit{vulnerability} etc. After extracting a possible list of commits, it should perform an
evaluation process to identify the actual commits that fixed security vulnerability.	This could be
hard because not all open source software repositories are using the same programming language.
Hence, it might be difficult to determine the actual lines of code that addressed the
vulnerabilities.

\section{Proposed Method}
Build a command-line interface program that is able to run two separate process: the mining process
and the evaluation process. The \textbf{mining} process takes a Git repository as input, searches
through the commit log, and return the list of commits that might potentially contain a patch as a
log file (JSON, HTML, etc.). The \textbf{evaluation} process takes a log file as input, and check
the code difference of every commit in the log file to identify the real patches.

\section{Tools}
\begin{itemize}
	\item PyGithub is a Python library build to access the GitHub API \cite{pygithub}.
	\item GitPython is a Python library build to interact with Git repositories using a combination of
	python and git command implementation \cite{gitpython}.
	\item Secbench Mining Tool is a repository mining tool build by The Quasar Research Group to mine
	vulnerability patterns from GitHub repositories \cite{secbench}.
\end{itemize}

% \chapter{Design}

% \chapter{Implementation and Testing}

% \chapter{Results and Discussion}

% \chapter{Conclusion}

\printbibliography[heading=bibintoc]

\end{document}